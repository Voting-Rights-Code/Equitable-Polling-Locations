\documentclass[11pt]{article}


\usepackage{amssymb, amsmath, verbatim, amsthm,url, multirow,fullpage,mathtools, appendix, mathrsfs, graphicx, outlines, subcaption}
\usepackage{longtable, rotating,makecell,array}
\usepackage[aligntableaux=top]{ytableau}


\setlength{\parindent}{0pt}
\setlength{\parskip}{1.5ex plus 0.5ex minus 0.2ex}


%***************************
%Frontmatter Table of contents
%***************************
% Annotations
%Equation display shortcuts
%Theorem environments
%***************************

%*****************
% Annotations
\usepackage{soul}
\usepackage[colorinlistoftodos,textsize=footnotesize]{todonotes}
\newcommand{\hlfix}[2]{\texthl{#1}\todo{#2}}
\newcommand{\hlnew}[2]{\texthl{#1}\todo[color=green!40]{#2}}
\newcommand{\sanote}{\todo[color=violet!30]}
\newcommand{\esnote}{\todo[color=orange!40]}
\newcommand{\note}{\todo[color=green!40]}
\newcommand{\newstart}{\note{The inserted text starts here}}
\newcommand{\newfinish}{\note{The inserted text finishes here}}
\setstcolor{red}
%***************************

%*****************
%Equation display shortcuts
\def\ba #1\ea{\begin{align} #1 \end{align}}
\def\bas #1\eas{\begin{align*} #1 \end{align*}}
\def\bml #1\eml{\begin{multline} #1 \end{multline}}
\def\bmls #1\emls{\begin{multline*} #1 \end{multline*}}
%*****************

%*****************
%Theorem environments
\newtheorem{thm}{Theorem}[section]
\newtheorem{conj}[thm]{Conjecture}
\newtheorem{lem}[thm]{Lemma}
\newtheorem{cor}[thm]{Corollary}
\newtheorem{prop}[thm]{Proposition}
\newtheorem{alg}[thm]{Algorithm}

\theoremstyle{remark}
\newtheorem{eg}[thm]{Example}
\newtheorem{claim}[thm]{Claim}

\theoremstyle{definition}
\newtheorem{dfn}[thm]{Definition}
\newtheorem{rmk}[thm]{Remark}
\newtheorem{ntn}[thm]{Notation}
%*****************

\title{South Carolina County Analysis}
\author{Voting Rights Code}
\date{today}
\begin{document}

\section{Summary}
	
	Counties: Berkeley, Greenville, Lexington, Richland, York.
\subsection{Data and Methodology}
	For all counties studied, we assign the voting age population to polls by 
\begin{outline}
	\1 Minimizing a function of the average distance and the maximum distance from the assigned polling location of the entire county's population.
		\2 5 people living a mile each from their assigned poll is prefered to 4 people being assigned a poll .5 miles away and the last being assigned one 3 miles away.
	\1 Requiring that each polling location have roughly the same number of people assigned to them
		\2 If a county has 2000 people and 10 polling locations, then each polling location cannot be assigned more than 30 people ($(2000 / 10) \times 1.5$ )
	\1 Population is assigned at the census block level (not at the individual level)
		\2 The entire voting age population is used, not registered voters
		\2 Data is from the 2020 census for all graphs
		\2 Racial/Ethnicity data is taken from the census.
			\3 Everyone is either Hispanic or Non-Hispanic
			\3 Everyone is in some racial category: White, Black, Asian, Pacific Islander, Other, Multi-Racial
			\3 This means that someone who is both Asian and Hispanic appear in both the Hispanic category and the Asian category.
\end{outline}

The results presented here are within 2$\%$ of optimal. By \emph{optimal} we mean that any other assignment of people to polling locations will either not satisfy these constraints, or will have higher average distance or greater inequality (i.e. standard deviation in distance to polls) than this assignment. 

We recognize that the constraints that we have implemented are overly simplistic, and that in actual poll assignment, there are more constraints that must be considered (such as precinct requirements which may not respect census block boundaries, or the fact that all adults in a county are not registered to vote).

We note that there are no demographic considerations in the model. We assign people to polls based only on the census block they live in. All demographic differences shown below are a result of the self segregated nature of our societies.

\subsection{Analysis overview}

\section{Demographic level distance to poll analysis \label{sec:distances}}
In this section, we present results on the distances to the assigned polls for the population as a whole and by the following demographics: White, Black, Asian (non-Pacific Islander) Native American and Hispanic.

We present data for both the average distance to the assigned polling location, and the equity weighted distance (which is the function we optimize for). We note that the equity weighted distance is always greater than the average distance (it can only be equal if everyone was located the \emph{the exact same distance} from the assigned poll, and it can never be less than the average distance). To get a sense of relative inequality experienced within a sub population, one may divide the equity weighted distance by the averaged distance. 
\begin{eg}
	For instance, if a the White population has an average distance of 10 and an equity weighted distance of 20, then while the African American population has an average distance of 7 and an equity weighted distance of 21, then we see that the distribution of distances to assigned polls has a greater spread for the African American population than for the White population ($21/7 = 3$ for the African American population while $20/10$ is 2 for the White.) This is not a measure of standard deviation of the distances, by any means. It is however, a quick way to get a rank order of the within population inequality for distance to the assigned poll.
\end{eg}

\subsection{Berkeley \label{sec:Berkeley distances}}
According to the census, in 2020, Berkeley county had a total of 173,949 adults, broken down by percent into the following ethnic and racial categories:

\begin{tabular} {| c | c |} 
	\hline
	Asian (non PI) &  0.025 \\ \hline
	African American & 0.22 \\ \hline
	Latine & 0.075 \\ \hline
	First Nations & 0.0067 \\ \hline
	White  & 0.64 \\ \hline
\end{tabular}

Because the First Nations population is so small, we do not include them in the the following discussion.

Their average and equity weighted distances to the polls from 2014-2022 are shown in Figure \ref{fig:Berkeley distance graphs}.

\begin{figure}
	\begin{subfigure}{.8\textwidth}
		\centering
		\includegraphics[width=.8\linewidth]{result analysis/Berkeley_SC_original_configs/orig_pop_scaled_avg}
		\label{sfig:Berkeley avg dist}
	\end{subfigure} \newline
	\begin{subfigure}{.8\textwidth}
		\centering
		\includegraphics[width=.8\linewidth]{result analysis/Berkeley_SC_original_configs/orig_pop_scaled_y_EDE}
		\label{sfig:Berkeley equity dist}
	\end{subfigure}
	\caption{Average and equity weighted distance for Berkeley county by demographic, 2014-2022}
	\label{fig:Berkeley distance graphs}
\end{figure}


From 2014-2020, the number of polling locations steadily increased from 48 to 60. In 2022, there were 36. During the first period, the distance from each census block to the assigned polling location decreased steadily from 2.4 kilometers to 1.9 kilometers. In 2022, this average distance jumps to 3.6 kilometers. 

Throughout the entire period, the majority White community is assigned to polling locations that are closer, on average than the African American community. In 2014, the majority White population's polling locations on average 322 meters closre than the African American community's locations. By 2020, this difference decreases to 237 meters. \sanote{comment about in spite of white community being more suburban?} However, in 2022, this difference jumps to 404 meters. Furthermore, in 2022, the community closest on average to their polling locations (White, 3.5 km) is over 350 meters further away from the worst affected community in any of the other years (African American, 2022). Also, while the minority Latine and Asian communities are closer to their polling locations than the White community during 2014-2020, they are further away in 2022.

On the other hand, the within group inequalities for all demographic groups is lower in 2022 than it is in any other year. While the average distances for all demographic groups decrease from 2014 to 2020, there is no consistent pattern for the within group inequalities.

The year 2020, had the largest number of polls and the lowest average distance. Compared to 2020, in 2022, the distance to assigned polling locations for the White community increases the least, 1.6 kilometer. The differences between 2022 and 2020 for various demographic groups is caputured in the table below:

\begin{tabular}{|c|c|}
	\hline
	Demographic group & Extra average meters in 2022 \\ \hline
	Asian (not PI) &   2362 \\ \hline
	African American &   1754  \\ \hline
	Latine & 2047 \\ \hline
	White &  1578\\ \hline
	Total population &  1689\\ \hline
\end{tabular}


\subsection{Greenville \label{sec:Greenville distances}}
According to the census, in 2020, Greenville county had a total of 406,243 adults, broken down by percent into the following ethnic and racial categories:

\begin{tabular} {| c | c |} 
	\hline
	Asian (non PI) &  0.025 \\ \hline
	African American & 0.16 \\ \hline
	Latine & 0.094 \\ \hline
	First Nations & 0.0044 \\ \hline
	White  & 0.69 \\ \hline
\end{tabular}

Because the First Nations population is so small, we do not include them in the the following discussion.

Their average and equity weighted distances to the polls from 2014-2022 are shown in Figure \ref{fig:Greenville distance graphs}.

\begin{figure}
	\begin{subfigure}{.8\textwidth}
		\centering
		\includegraphics[width=.8\linewidth]{result analysis/Greenville_SC_original_configs/orig_pop_scaled_avg}
		\label{sfig:Greenville avg dist}
	\end{subfigure} \newline
	\begin{subfigure}{.8\textwidth}
		\centering
		\includegraphics[width=.8\linewidth]{result analysis/Greenville_SC_original_configs/orig_pop_scaled_y_EDE}
		\label{sfig:Greenville equity dist}
	\end{subfigure}
	\caption{Average and equity weighted distance for Greenville county by demographic, 2014-2022}
	\label{fig:Greenville distance graphs}
\end{figure}


From 2014-2018, there were 150 to 151 polling locations. In 2020, there were 145, and in 2020 there were 106. During the first period, on average, the population is assigned to a polling location 1.3 kilometers from their census block, in 2022, this average distance jumps to 1.4 kilometers and in 2022, it jumps again to 1.8.

While the average distance from the assigned polling locations jump slightly from the first three elections to 2020, the equity weighted distance remains very constant during all four years, indicating that while a reduction in the number of polling locations increased average distance to the polls, the new locations were placed more equitably than they had been in the previous three elections. By contrast, in 2022, the equity weighted distance jumped nearly 700 meters for the county overall. 

We note that the majority White populations are assigned to polling locations further away than the minority Black population. In 2014-2020, the White population is assigned a poll about 230-250 meters further away than the African American community. \hlfix{Some of this is due to the White population living in less dense areas where distances are larger in general}{verify}, see Section \ref{????}. However, in 2022, the average distances differ by over 290 meters. Furthermore, in 2022, the community closest on average to their polling locations (Latine, 1.5 km) is over 80 meters further away from the worst affected community in any of the other years (White, 2020).

In terms of within group inequality, from 2014-2020, the demographic communities are consistently ranked (interms of increasing inequality) African America, Asian, Latine then White. In 2022, however, the African American community experiences within group inequality similar to the Asian community in the previous period. The Asian community experiences within group inequality similar to the Latine community in the previous period, the Latine community experiences the greatest inequality across all 5 years while the White within group inequality does not siginificantly change. 

The year 2020, had the largest number of polls and the lowest average distance. Compared to 2020, in 2022, the African American and White communities were assigned to polling locations the furthest away. The differences between 2022 and 2020 for various demographic groups is caputured in the table below:

\begin{tabular}{|c|c|}
	\hline
	Demographic group & Extra average meters in 2022 \\ \hline
	Asian (not PI) &   334 \\ \hline
	African American &   363  \\ \hline
	Latine & 301 \\ \hline
	White &  405\\ \hline
	Total population &  389\\ \hline
\end{tabular}

\subsection{Lexington \label{sec:Lexington distances}}
According to the census, in 2020, Lexington county had a total of 22,504  adults, broken down by percent into the following ethnic and racial categories:

\begin{tabular} {| c | c |} 
	\hline
	Asian (non PI) &  0.022 \\ \hline
	African American & 0.14 \\ \hline
	Latine & 0.061 \\ \hline
	First Nations & 0.0048 \\ \hline
	White  & 0.75 \\ \hline
\end{tabular}

Because the First Nations population is so small, we do not include them in the the following discussion.

Their average and equity weighted distances to the polls from 2014-2022 are shown in Figure \ref{fig:Lexington distance graphs}.

\begin{figure}
	\begin{subfigure}{.8\textwidth}
		\centering
		\includegraphics[width=.8\linewidth]{result analysis/Lexington_SC_original_configs/orig_pop_scaled_avg}
		\label{sfig:Lexington avg dist}
	\end{subfigure} \newline
	\begin{subfigure}{.8\textwidth}
		\centering
		\includegraphics[width=.8\linewidth]{result analysis/Lexington_SC_original_configs/orig_pop_scaled_y_EDE}
		\label{sfig:Lexington equity dist}
	\end{subfigure}
	\caption{Average and equity weighted distance for Lextington county by demographic, 2014-2022}
	\label{fig:Lexington distance graphs}
\end{figure}

From 2014-2022, there were 92-95 polling locations. This is the one county that did not see a drop in polling locations in 2022. During the entire period, the population is assigned to a polling location 1.7 kilometers from their census block, with the lowest average distance from 2014-2018 when there were 95 polls (1708 meters), and the largest in 2020 when there were the fewest number of polls (1733 m).

The majority White population consistently lives 190 to 200 meters further to their polls than the minority Black community. \hlfix{Some of this is due to the White population living in less dense areas where distances are larger in general}{verify}, see Section \ref{????}.

The within group inequality is consistent across the set of 5 years, with the minority Asian community faring best, following by the majority White community, then the African American and Latine communities. 

As a result, we do not see much change in either the average distances or equity weighted distances for the different communities from 2014 - 2022. 


\subsection{Richland \label{sec:Richland distances}}
According to the census, in 2020, Richland county had a total of 327,481  adults, broken down by percent into the following ethnic and racial categories:

\begin{tabular} {| c | c |} 
	\hline
	Asian (non PI) &  0.028 \\ \hline
	African American & 0.44 \\ \hline
	Latine & 0.056 \\ \hline
	First Nations & 0.0034 \\ \hline
	White  & 0.45 \\ \hline
\end{tabular}

Because the First Nations population is so small, we do not include them in the the following discussion.

Their average and equity weighted distances to the polls from 2014-2022 are shown in Figure \ref{fig:Richland distance graphs}.

\begin{figure}
	\begin{subfigure}{.8\textwidth}
		\centering
		\includegraphics[width=.8\linewidth]{result analysis/Richland_SC_original_configs/orig_pop_scaled_avg}
		\label{sfig:Richland avg dist}
	\end{subfigure} \newline
	\begin{subfigure}{.8\textwidth}
		\centering
		\includegraphics[width=.8\linewidth]{result analysis/Richland_SC_original_configs/orig_pop_scaled_y_EDE}
		\label{sfig:Richland equity dist}
	\end{subfigure}
	\caption{Average and equity weighted distance for Richland county by demographic, 2014-2022}
	\label{fig:Richland distance graphs}
\end{figure}

From 2014-2020, there were 140 to 143 polling locations. In 2022, there were 111. During the first period, on average, the population is assigned to a polling location 1.3 kilometers from their census block, while in 2022, this average distance jumps to 1.5 kilometers. 

We note that the plurality White and African American populations are assigned to polling locations similar distances away. In 2014-2020, the average distances for the populations are less than 30 meters of each other, with the African American community being assigned generally closer than the White community. \hlfix{Some of this is due to the White population living in less dense areas where distances are larger in general}{verify}, see Section \ref{????}. However, in 2022, the average distances differ by over 40 meters with the African American community traveling, on average, further. 

While the average distance increased for everyone from the 2014-2020 period to 2022, the equity weighted distances in did not change very much at all for the White or Latine communities, but did drastically for the Asian or African American communities. 

In terms of within group inequality, the 2022 polls gave the minority Asian community the best equity of any demographic over the entire time period. The White and Latine communities also experienced less within group inequality in 2022 than they had in any other year. The withing group inequality for the African American community remained effectively unchanged. 

Note that in 2020, in spite of having the fewest number of poling locations prior to 2022 (140), the average distance traveled by all demographic groups was the least. Compared to 2020, in 2022,
the African American community was assigned to a polling locations the furthest away (on average almost a kilometer further away), in spite of being on consistently the closest to their polls prior to 2020. The differences between 2022 and 2016 for various demographic groups is caputured in the table below:

\begin{tabular}{|c|c|}
	\hline
	Demographic group & Extra average meters in 2022 \\ \hline
	Asian (not PI) &   192 \\ \hline
	African American &   239  \\ \hline
	Latine & 161 \\ \hline
	White &  160\\ \hline
	Total population &  201\\ \hline
\end{tabular}

\subsection{York \label{sec:York distances}}
According to the census, in 2020, York county had a total of 213,111 adults, broken down by percent into the following ethnic and racial categories:

\begin{tabular} {| c | c |} 
\hline
 Asian (non PI) &  0.029 \\ \hline
African American & 0.18 \\ \hline
Latine & 0.0572 \\ \hline
First Nations & 0.0074 \\ \hline
White  & 0.70 \\ \hline
\end{tabular}

Because the First Nations population is so small, we do not include them in the the following discussion.

Their average and equity weighted distances to the polls from 2014-2022 are shown in Figure \ref{fig:York distance graphs}.

\begin{figure}
	\begin{subfigure}{.8\textwidth}
		\centering
		\includegraphics[width=.8\linewidth]{result analysis/York_SC_original_configs/orig_pop_scaled_avg}
		\label{sfig:York avg dist}
	\end{subfigure} \newline
	\begin{subfigure}{.8\textwidth}
		\centering
		\includegraphics[width=.8\linewidth]{result analysis/York_SC_original_configs/orig_pop_scaled_y_EDE}
		\label{sfig:York equity dist}
	\end{subfigure}
	\caption{Average and equity weighted distance for York county by demographic, 2014-2022}
	\label{fig:York distance graphs}
\end{figure}

From 2014-2020, there were 88 or 89 polling locations. In 2022, there were 79. During the first period, on average, the population is assigned to a polling location between 1.5 and 1.6 kilometers from their census block, while in 2022, this average distance jumps to 1.7 kilometers. 

We note that the majority White population generally has to travel longer distances than the minority African American population, but as we show in Section \ref{???}, this is largely due to the White population living in less dense areas where distances are larger in general. 

In terms of the within group inequality, the Asian population is consistently has the most uniform distance to the assigned poll. The greatest within group inequalities are experiences (in decreasing order) by the White, Latine, African American communities in 2022.

Note that in 2016, inspite of having the fewest number of poling locations prior to 2022 (89), the average distance traveled by all demographic groups was the least. Compared to 2016, in 2022, the African American community was assigned to a polling locations the furthest away (on average half a kilometer further away), in spite of being on consistently the closest to their polls prior to 2020. The differences between 2022 and 2016 for various demographic groups is caputured in the table below:

\begin{tabular}{|c|c|}
	\hline
	Demographic group & Extra average meters in 2022 \\ \hline
	Asian (not PI) &   102 \\ \hline
	African American &   553  \\ \hline
	Latine & 170 \\ \hline
	White &  162\\ \hline
	Total population &  232\\ \hline
\end{tabular}
	
\end{document}